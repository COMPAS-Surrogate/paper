% Define document class
\documentclass[twocolumn]{aastex631}
\usepackage{showyourwork}

% Begin!
\begin{document}

% Title
\title{An open source scientific article}

% Author list
\author{@COMPAS-Surrogate}

% Abstract with filler text
\begin{abstract}
    Lorem ipsum dolor sit amet, consectetuer adipiscing elit.
    Ut purus elit, vestibulum ut, placerat ac, adipiscing vitae, felis.
    Curabitur dictum gravida mauris, consectetuer id, vulputate a, magna.
    Donec vehicula augue eu neque, morbi tristique senectus et netus et.
    Mauris ut leo, cras viverra metus rhoncus sem, nulla et lectus vestibulum.
    Phasellus eu tellus sit amet tortor gravida placerat.
    Integer sapien est, iaculis in, pretium quis, viverra ac, nunc.
    Praesent eget sem vel leo ultrices bibendum.
    Aenean faucibus, morbi dolor nulla, malesuada eu, pulvinar at, mollis ac.
    Curabitur auctor semper nulla donec varius orci eget risus.
    Duis nibh mi, congue eu, accumsan eleifend, sagittis quis, diam.
    Duis eget orci sit amet orci dignissim rutrum.
\end{abstract}

% Main body with filler text
\section{Introduction}
\label{sec:intro}

Lorem ipsum dolor sit amet, consectetuer adipiscing elit.
Ut purus elit, vestibulum ut, placerat ac, adipiscing vitae, felis.
Curabitur dictum gravida mauris, consectetuer id, vulputate a, magna.
Donec vehicula augue eu neque, morbi tristique senectus et netus et.
Mauris ut leo, cras viverra metus rhoncus sem, nulla et lectus vestibulum.
Phasellus eu tellus sit amet tortor gravida placerat.
Integer sapien est, iaculis in, pretium quis, viverra ac, nunc.
Praesent eget sem vel leo ultrices bibendum.
Aenean faucibus, morbi dolor nulla, malesuada eu, pulvinar at, mollis ac.
Curabitur auctor semper nulla donec varius orci eget risus.
Duis nibh mi, congue eu, accumsan eleifend, sagittis quis, diam.
Duis eget orci sit amet orci dignissim rutrum.

Nam dui ligula, fringilla a, euismod sodales, sollici- tudin vel, wisi.
Morbi auctor lorem non justo, nam lacus libero, pretium at, lobortis vitae.
Donec aliquet, tortor sed accumsan bibendum, erat ligula aliquet magna.
Morbi ac orci et nisl hendrerit mollis, suspendisse ut massa, cras nec ante.
Pellentesque a nulla cum sociis natoque penatibus et magnis dis parturient.
Aliquam tincidunt urna, nulla ullamcorper vestibulum turpis.
Pellentesque cursus luctus mauris \citep{Luger2021}.




\section{Introduction} \label{sec:intro}
This work builds of Riley et al 2022

Astrophysical processes that govern stellar, binary system evolution, and the production of binary black holes (BBH) that produce gravitational waves (GW) detectable by current instrumentation (e.g. aLIGO, VIRGO, KAGRA, the \lvk) are very uncertain. 
Over the next few years, we will accumulate a population of several hundred BBH mergers. 
We can use the population of mergers to constrain some parameters describing astrophysical processes, such as the star formation rate in the early universe. 
In this work, we use \compas, a rapid stellar population synthesis tool, to model the universe with different star formation rates (SFR) and try to place constraints on the SFR that can reproduce the \lvk\ detections. 


The remainder of the paper is organised as follows.
Section~\ref{sec:method} outlines...
Results are summarised in Section~\ref{sec:results}.
The data products, and software to reproduce the results available online as supplementary materials (\projectUrl).
Finally, we discuss caveats and provide concluding remarks in Section~\ref{sec:conclusion}.

\section{Method} \label{sec:method}


The \lvk\ has detected $\sim90$ GW signals from a population of merging BBHs. 
Each BBH has some parameters $\vec{\theta}$ that describe the system, such as the BBH's chirp-mass $\mathcal{M}$, spin magnitude $\chi$, spin angle $\theta$, and redshift $z$.
On the other hand, \compas\ can generate populations of merging BBHs using different initial star formation parameters $\vec{\lambda}$. 

Therefore, we can use \compas\ to determine the posterior on $\vec{\lambda}$ given \lvk's $N_{\rm obs}$ observed BBHs and their parameters $\vec{D} = \{ p_i(\vec{\theta}|h), ...\}$ (where $h$ is the strain data):
\begin{equation}
    p(\vec{\lambda}| \vec{D}) \sim \pi(\vec{\lambda})\ \mathcal{L}(\vec{D}| \vec{\lambda})\ .
\end{equation}


Here, $\mathcal{L}(\vec{D}| \vec{\lambda})$ can be written as:
\begin{equation}
    \ln \mathcal{L}(\vec{D}| \vec{\lambda}) = \ln \mathcal{L}(N_{\rm obs}|\vec{\lambda}) + \sum^{N_{\rm obs}}_{i=1} \ln \mathcal{L}(D_i|\vec{\lambda}) \ .
\end{equation}


Expected gains 




\section{Toy Model}

In this section, we delve into a toy-model problem of just using the \lvk\ chirp-masses and \compas\ chirp-masses, marginalising over the redshift. 

In this case, we create a surrogate for $\mathcal{L}$.





\section{Results}\label{sec:results}
Details on the results

\section{Discussion and caveats}\label{sec:conclusion}
Some concluding discussions

\section{Data and software availability}\label{sec:data}
Code data can be found at \projectGit



%%%%%%%%%%%%%%%%%%%%



\section*{Acknowledgments}{


We gratefully acknowledge the Swinburne Supercomputing OzSTAR Facility for computational resources. All analyses (including test and failed analyses) performed for this study used $XX$K core hours on OzSTAR. This would have amounted to a carbon footprint of ${\sim XX{\text{t CO}_2}}$~\citep{greenhouse, energy_to_co2_converter}. Thankfully, as OzSTAR is powered by wind energy from Iberdrola Australia; the electricity for computations produces negligible carbon waste.


A.V. is supported by the Australian Research Council (ARC) Centre of Excellence CE170100004.

}

\vspace{5mm}
\facilities{LIGO-VIRGO-KAGRA}

\software{
\python~\citep{pythonForScientificComputing,pythonForScientists},
\astropy~\citep{astropy},
\arviz~\citep{arviz_2019},
\exoplanet~\citep{Foreman-Mackey:2021:JOSS},
\lightkurve~\citep{LightkurveCollaboration:2018:ascl},
\starry~\citep{Luger:2019:AJ},
\celerite~\citep{Foreman-Mackey:2017:ascl},
\pymc~\citep{Salvatier:2016:ascl},
\numpy~\citep{numpy},
\scipy~\citep{SciPy},
\pandas~\citep{pandas},
\matplotlib~\citep{matplotlib},
\corner~\citep{corner},
\sphinx~\citep{sphinx_doc},
\jupyter~\citep{jupyter},
\jupyterbook~\citep{jupyter_book}.
}





\bibliography{bib}

\end{document}
